%One of the best resources for \LaTeX basics, and advanced constructs, is the \LaTeX wikibook\footnote{To be found at~\url{http://en.wikibooks.org/wiki/LaTeX/}}. Of course fellow students, colleagues and a good internet search using your favorite search engine can do wonders if you're stuck. 

\section{Libusb}\label{sec:libusb}
Libusb on universaalne teek, kirjutatud C keeles, et hõlbustada programmide loomist, mis suhtlevad USB seadmetega.

