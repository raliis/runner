Projekti raames mõeldud projekt lõpuni ei jõudnud, kuna tekkisid takistused, mida ei jõutud olemasoleva ajaga lahendada.

\section{Takistused}
Suurim takistus oli saada ettekujutus sellest, kuidas kell andmeid välja annab.
Polar kelladel tundub ka olema igal erinev viis andmete väljastamiseks \cite{flowlink-git}, mille tõttu on programmi loomine vastavalt igale kellale eraldi väljakutse.

Kella kohta dokumentatsiooni leidmine osutus loodetust keerulisemaks.
Võimalikke põhjuseid selleks on mitmeid, suurim neist, et antud seade on vana.
Sellele lisaks ei ole ka ettevõte, kes selle kella tootnud on, näinud väärtust lasta teistel rakendusi luua kogutud andmete analüüsiks
!!! kas see on common sense? vist mitte? !!!

\section{Arendusega seotud otsused}
Programmi luues oleks saanud ka esimesena teha valmis liides, millega oleks saanud enamuse programmi funktsioonidest ära luua.
Selle lahenduse juures oleks saanud kasutada manuaalselt sisestatud andmeid ja teha töötav programm, mida oleks pidanud hiljem ümber kohandama.
Seda rada ei valitud, kuna liides ise ja andmetöötlus on projekti vähemtähtis osa.
Selle otsusega suunati kogu arendustöö just tehnilise \textit{back-end} poole peale.

