Põhiline eesmärk projekti raames oli saada kellast kätte andmed, mida seejärel analüüsima hakata.
Plaan oli, et kui andmed on käes minnakse edasi programmi ehitamisega, kuid selle osani ei jõutud, kuna kellast andmete kättesaamine osutus liiga keeruliseks ja aeganõudvaks.

\section{Takistused}
Suurim takistus oli saada ettekujutus sellest, kuidas kell andmeid välja annab.
Uurides olemasolevaid lahendusi jäi mulje, et Polar kelladel on igal mudelil erinev viis andmete väljastamiseks, mille tõttu on programmi loomine vastavalt igale kellale eraldi väljakutse.\cite{flowlink-git}

Kella kohta dokumentatsiooni leidmine osutus loodetust keerulisemaks.
Võimalikke põhjuseid selleks on mitmeid, suurim neist, et antud seade on vana.
Sellele lisaks ei ole ilmselt huvi ettevõttel väljastada enda kellade kohta dokumentatsiooni, kuna rakendus oli nende enda toodetud ja kolmandate osapoolte tarkvarade kasutamine kella väljalaske ajal oleks olnud ebasoovitav.

Potentsiaalne takistus millega ka alguses juba arvestati, kuid milleni ei jõutud on teadmatus, mis kujul andmed väljastatakse.
Kui need tekstina väljastataks oleks nende töötlemine lihtne, kuid kui andmed oleks kas krüpteeritud või mõnes tundmatus formaadis, oleks see protsess omakorda veel raskem.

\section{Arendusega seotud otsused}
Programmi luues oleks saanud ka esimesena teha valmis liides, millega oleks saanud enamuse programmi funktsioonidest ära luua.
Selle lahenduse juures oleks saanud kasutada manuaalselt sisestatud andmeid ja teha töötav programm, mida oleks pidanud hiljem ümber kohandama.
Seda rada ei valitud, kuna liides ise ja andmetöötlus on projekti vähemtähtis osa.
Selle otsusega suunati kogu arendustöö just tehnilise \textit{back-end} poole peale.

