
Projektis kasutatud kellal on ainus liides USB, Bluetooth võimekus puudub.
Kasutusel olev USB versioon on 2.0.

USB seadmetega suhtlus toimub mööda kontseptuaalseid torusid.
Iga toru on ühendustee rakenduse ja seadme \textit{endpointi} vahel.
Torud jagunevad omakorda kaheks: sõnumite ja voo torud.
Sõnumite torusid kasutatakse kontrolledastusteks, mis on lühikesed ja lihtsad käsklused seadmele, voo torusid kasutatakse andmeedastuseks.\cite{dreamincode}
Seadme \textit{endpoint} on riistvaraline osa seadmest, tavaliselt kogumik registreid kindla eesmärgiga, näiteks korralduste vastuvõtmine või andmete ülekandmine.\cite{endpoints}
Igal USB seadmel on alati olemas \textit{endpoint} 0, mida kasutatakse põhiliste käskude edastamiseks ja seadme konfiguratsiooni seadmiseks seadme ühendamisel.


\section{Libusb}\label{sec:libusb}
Libusb on C keeles kirjutatud universaalne teek, mis on loodud tarkvaraarendajatele USB seadmetega suhtlevate programmide kirjutamise hõlbustamiseks.
See annab väga täpse kontrolli rakenduse suhtluse üle seadmega ning toetab mitmeid USB versioone.
Lisaks ei sõltu teek konkreetsest platvormist, nii et seda saab kasutada mitmel operatsioonisüsteemil töötava rakenduse loomisel.
Antud teek sai valitud, kuna sobis täpselt probleemi lahendamiseks kuna võimaldas piisavalt täpset kontrolli programmi töö üle.\cite{libusb}
Teek koosneb moodulitest, millest igaüks on erineva otstarbega.

\subsection{Kasutus}
Teegi kasutamiseks tuleb see esimese sammuna initsialiseerida.
Seejärel tuleb leida arvuti kügle ühendatud seadmete hulgast soovitud seade üles leida.
Kui seade on valitud, tuleb selle suhtlused teiste programmidega lõpetada ja siis suhtluseks avada.
Peale seadme edukat avamist on see võimeline vastu võtma programmilt andmeid ja käsklusi.
Suhtluseks seadmega on olemas kaks varianti: sünkroonse ja asünkroonse IO moodul.
Sünkroonse andmevahetuse APIt on lihtsam kasutada, kuid see pole nii võimas, kui asünkroonne.
Sünkroonse andmevahetuse suurim puudujääk on, et kogu suhtlus toimub sama lõime peal, kus ka ülejäänud programmi töö, pannes seega programmi töö ootama, kuniks andmevahetus läbi on.
Asünkroonne suudab suhtluse seadmega taustale suunata, et kasutajaga suhtlev programm saaks edasi funktsioneerida sõltumata sellest, kui kaua toimub andmevahetus seadmega. 


\section{Hidapi}
Hidapi on teek, mis on loodud HID seadmetega suhtlust hõlbustama, realiseerides ära enim kasutatud funktsioonid, kaasaarvatud andmete saatmine ja vastuvõtmine.
Hidapil on igal platvormil erinev \textit{back-end}, Linuxi platvormil on neid valikus 2.
Esimene on Linuxi kernelis sees olev \textit{hidraw} draiver ja teine on libusb teegiga realiseeritud versioon.
Projekti kirjutamisel on juba kasutusel libusb teek ning seega valiti selle \textit{back-end}iga versioon, et veelgi lihtsustada programmi kirjutamise protsessi.
Teeki on võimalik väga lihtsalt ühe \textit{header} failina programmi kaasata.\cite{hidapi}
