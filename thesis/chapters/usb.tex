%One of the best resources for \LaTeX basics, and advanced constructs, is the \LaTeX wikibook\footnote{To be found at~\url{http://en.wikibooks.org/wiki/LaTeX/}}. Of course fellow students, colleagues and a good internet search using your favorite search engine can do wonders if you're stuck. 

\section{Libusb}\label{sec:libusb}
Libusb on universaalne teek, kirjutatud C keeles, et hõlbustada programmide loomist, mis suhtlevad USB seadmetega.

Kell suhtleb arvutiga kasutades USB ühendust.

USB standard klassifitseerib ära seadmete klassid. Antud kell klassifitseerub HID seadmeks, mis tähendab seda, et see on seade millega inimesed saavad otseselt kasutada.
HID seadmete jaoks on ette nähtud omaette suhtlusviis, kuidas saab edastada teateid seadmele ja sealt infot kätte.

kirjelda kuidas see suhtlus on et koigepealt avad jne 

siis kuidas see muu osa seal on et ei taha seda surnuks pommitada

mis asjad on endpointid jne

