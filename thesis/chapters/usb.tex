%One of the best resources for \LaTeX basics, and advanced constructs, is the \LaTeX wikibook\footnote{To be found at~\url{http://en.wikibooks.org/wiki/LaTeX/}}. Of course fellow students, colleagues and a good internet search using your favorite search engine can do wonders if you're stuck. 

Projektis kasutatud kellal bluetooth võimekus puudub, seega ainus võimalus andmeid kätte saada on USB kaudu.
USB standard klassifitseerib ära seadmete klassid. Antud kell klassifitseerub HID seadmeks, mis tähendab seda, et see on seade millega inimesed saavad otseselt kasutada.
HID seadmete jaoks on ette nähtud omaette suhtlusviis, kuidas saab edastada teateid seadmele ja sealt infot kätte.


\section{Libusb}\label{sec:libusb}
Libusb on universaalne teek, kirjutatud C keeles, et hõlbustada programmide loomist, mis suhtlevad USB seadmetega.


\section{Hidapi}
Hidapi on teek, mis on konkreetsemalt mõeldud HID seadmetega suhtlema.

[need notes on this, mida siin tapselt oelda]

mis asjad on endpointid jne

kirjelda kuidas see suhtlus on et koigepealt avad jne 
