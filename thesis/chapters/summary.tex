Käesoleva lõputöö eesmärk oli luua lihtne ja minimaalne andmeliides Polar RC3 GPS spordikellale, mis suudaks sealt saada kätte kasutaja treeningandmed ja võimaldaks kasutajal siis neid analüüsida.

Projekti käigus prooviti tutvuda kella tööpõhimõttega ning sellele toetudes kirjutada tarkvara.
Kella riistvaraga tutvuti kell lahti võttes ja visuaalselt vaadeldes, komponente lahti emaplaadilt ei võetud.
Andmeedastust prooviti uurida kasutades olemasolevat tootjapoolset programmi.
Leitud alaoogset lahendust originaalsega prooviti kasutada ja ka kohandada soovitud tulemuse saavutamiseks, kuid lõplik lahendus oli sobiva tarkvara ise kirjutamine, mis jäi poolikuks.

Andmete kellast kättesaamine osutus liiga keeruliseks ja sellega seoses olid liiga suured takistused, mida selle projekti raames ei suudetud kõrvaldada.
Põhiliseks takistuseks osutus kellaga suhtlemiseks õige viisi leidmine, kuna dokumentatsioon puudus ja originaalne rakendus enam ei tööta.
Programm on valminud kuni seadme avamiseni ja andmeid seadmest kätte ei saa.

Tulevikus tuleks jätkata programmi arendamist, et saada kätte sobiv lahendus, mis võimaldaks kasutada täielikult ka vanemaid seadmeid, millele enam toetust tootja poolt pole.