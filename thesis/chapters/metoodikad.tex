Kellast andmete pärimine on teadmata protsess antud projekti raames.
Andmetele ligipääsu saamiseks on erinevaid metoodikaid, mida ka allpool kirjeldatatakse ja ka põhjendatakse, miks konkreetne valik tehti.

\section{Originaalse rakenduse jäljendamine}
Lihtsaim variant oleks ühendada kell arvuti külge, millel töötab tootja enda originaalne rakendus ning kasutada seda andmete kättesaamiseks, samal ajal USB liini jälgides.
USB liini jälgimiseks saab kasutada nii tarkvara, nt Wireshark, kui ka riistvaralist lahendust.
Kui on käes suhtlus kella ja arvuti vahel, saab saadud tulemust proovida jäljendada.
Sellise tegutsemise juures on ka oht kella kahjustada kõige väiksem, kuna seda peab vähem ebavajalike päringutega "pommitama".
Kuna originaaltarkvara töötas vaid Mac ja Windows keskkondades, on selleks vaja töötavat arvutit sellise operatsioonisüsteemiga.

\section{Sarnase projekti jäljendamine}
Keerukam lahendus on leida mõni sarnane projekt ja tutvuda sellega, kuidas antud programm töötab.
Kui leitud tarkvara toimib, saab proovida seda jäljendada.
Kui see ei toimi, tuleb seda proovida kohandada, et see tööle hakkaks.
Selline lähenemine hõlmab palju katsetamist, mis ei ole optimaalne.
Katsetamise kaks suuremat miinust on ajakulu ja ka võimalus, et saates kellale suvalisi päringuid, võib üle kirjutada kätte saada tahetavaid andmeid, kui ka suuremat kahju teha.

\section{Valitud metoodika}
Esimene valik projekti alustades oli originaalse rakenduse jäljendamise tee, kuna see tundus kõige lihtsam ja selgem viis probleemi lahendamiseks.
Selle proovimise käigus selgus, et nii konkreetse seadme toetus, kui ka serveripoolne teenus, mis võimaldas vaadata ja hoiustada andmeid on töötamise lõpetanud ja kasutajatele enam kättesaamatu.
Rakendus andmeid kasutaja seadmesse ei salvestanud, vaid suunas need otse pilve, mille tõttu ei olnud enam võimalik kasutada seda rakendust ka lihtsalt andmete saamise protsessi õppimise jaoks.


