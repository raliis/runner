
\section{Tootjapoolne tarkvara}\label{sec:tootja-soft}
Polari tarkvara, millega sai kellast andmeid tõmmata, oli Polar WebSync.
Selle tööpõhimõte oli järgnev:
Tuli ühendada kell arvuti külge, seejärel käivitada rakendus.
Andmete kättesaamiseks ja vaatamiseks oli vajalik kasutajakonto nende veebikeskkonnas.
Kui konto oli olemas, tuli logida sisse enda kasutajanime ja parooliga rakendusse, mis seejärel tõmbas andmed kellast veebikeskkonda polarpersonaltrainer.com, ilma kasutaja masinasse andmeid salvestamata, kus neid siis töödeldi ja seejärel sai andmeid veebilehitsejas sisselogituna vaadata. 
Aasta 2019 lõpus lõpetati toetus ära rakendusele WebSync ja sulgeti ka veebikeskkond, kuna see oli vana ja sooviti keskenduda uue süsteemi toetamisele.\cite{polar-ws-discontinued}

Uus keskkond on tehniliselt piisavalt erinev vanast, et andmete üle kandmine vanast keskkonnast polnud võimalik.
Vanemad seadmed ning vana rakendus uue keskkonnaga ei liidestu.
Vanemate seadmete omanikele pakuti üleminekuperioodil, mis lõppes 29. veebruar 2019, ka soodustusi uuema riistvara ostmiseks.\cite{polar-ws-discontinued}
Selle muudatuse tagajärjel on inimesed, kes kasutavad ikkagi vana riistvara, jäänud ilma võimalusest oma treeningute andmeid analüüsida rohkem, kui kellast iga individuaalse treeningu andmeid vaadates.

Sellel rakendusel oli mitu negatiivset omadust.
Suurim neist oli, et kasutaja ei omanud oma andmeid ja pidi kasutama tootja enda keskkonda enda andmete nägemiseks, kuhu ilma registreerimata ligi ei pääsenud.
Eelnevaga kaasaskäiv miinus on, et andmete nägemiseks oli vajalik internetiühendus, kuna kõik andmed elasid pilves.
Lisaks eelnevalt mainitud puudujääkidele ei olnud rakendus saadaval Linux platvormil, mis tähendab et osad kasutajad pidid leidma mingi viisi kasutada kas Windows või Mac platvormi.
Lisaks negatiivsetele omadustele oli ka positiivne, et rakendus töötas hästi ja oli kergelt arusaadav kasutajale.


\section{Polar-Flowlink-linux}\label{sec:flowlink}
Projekti tegemise poole peal leiti ka githubist analoogne projekt kunagise WebSync tarkvaraga.
Leitud tarkvara nimi on Polar flowlink.
Flowlink koosneb mitmest osast:
Esimene osa tõmbab kellast andmed, teine suunab need andmebaasi ning kolmas osa oli veebiliides, millega on võimalik saadud andmeid vaadata.
Projekti kirjelduses oli ka kirjas, et see rakendus on kirjutatud Polar FT60 HRM jaoks.
Autor ka väitis et proovis programmi FT80 mudeliga ja selle peal see ei töötanud.
Seda projekti prooviti ka autori kellaga ning sellega samamoodi ei saadud andmeid kätte.

Leitud projekti puhul on suur eelis, et lähtekood on avatud ja siis saab sellega tutvuda, et saada aimu sellest, kuidas see töötama peaks.
Miinustena saab välja tuua aga sobimatuse seadmega, mis ka muudab kogu rakenduse mittekasutatavaks.
Andmete saamisele lisaks töötavad andmebaas ja veebiliides on ka antud rakenduse jaoks üleliigsed.


\section{Võrdlus}
Mõlemal olemasoleval ja leitud rakendusel on nii positiivseid külgi kui ka negatiivseid.
Kui mõlemad töötaksid ja ühilduksid kellaga, saaks neid kasutada küll andmete visualiseerimiseks ja analüüsimiseks, kuid 
Kuigi tootja enda tarkvara oli palju rohkemate omaduste ja funktsionaalsustega, kui leitud analoog, oleks see siiski ebamugav lahendus liiga paljude osadega.
Eraldiseisvat rakendust on vaja, et oleks olemas ka äärmiselt lihtne ja minimaalne liides andmete nägemiseks.
