
\section{Tootjapoolne tarkvara}\label{sec:tootja-soft}
Polari tarkvara, millega sai kellast andmeid tõmmata ja analüüsida, oli Polar WebSync.
Selle tööpõhimõte oli järgnev:
Tuli ühendada kell arvuti külge, seejärel käivitada rakendus.
Andmete kättesaamiseks ja vaatamiseks oli vaja luua kasutajakonto nende veebikeskkonda.
Kui konto oli olemas, tuli logida sisse enda kasutajanime ja parooliga rakendusse, mis seejärel tõmbas andmed kellast pilve, kus neid siis töödeldi ja seejärel sai andmeid veebilehitsejas sisselogituna vaadata. 
Kasutaja masinasse andmeid ei jäetud, nii et neist koopiat teha ei olnud võimalik.
Aasta 2019 lõpus\cite{polar-ws-discontinued} lõpetati toetus ära rakendusele WebSync ja sulgeti ka veebikeskkond polarpersonaltrainer.com, kuna see oli vana ja sooviti keskenduda uue süsteemi toetamisele.
Uus keskkond on tehniliselt erinev vanast, nii et andmete üle toomine ning vana rakenduse kasutamine ei olnud võimalik.
Selle tagajärjel on inimesed, kes kasutavad vana riistvara jäänud ilma võimalusest oma treeningute andmeid analüüsida rohkem, kui kellast iga individuaalse treeningu andmeid vaadates.

