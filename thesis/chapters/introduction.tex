%%Some basic ways to manipulate text are \textit{italics} and \textbf{bold}. One can reference Figures (see Figure \ref{fig:taltech} for an example) as well as cite references which are defined in the \textit{references.bib} file.\cite{spectre,example-reference} 
%The \textit{Bibliography}, \textit{List of Figures} and \textit{List of Tables} are all automatically generated and references will be updated automatically as well. This means that if you've defined a citation but are not referencing it, it will not appear in the \textit{Bibliography}. This also means that any Figure / Table / Citations numbers are automatically updated as well. Numbering is done by order-of-appearance.
%
%One can create an itemized list:
%\begin{itemize}
%    \item item a
%    \item item b
%    \item ...
%\end{itemize}
%
%Or enumerate them:
%\begin{enumerate}
%    \item item x
%    \item item y
%    \item ...
%\end{enumerate}
%
%
%\begin{figure}[ht]
%    \centering
%    \includegraphics[width=.5\textwidth]{figures/taltech.jpg}
%    \caption{\textit{An image of the TalTech logo.}}
%    \label{fig:taltech}
%\end{figure}
%
%
%A table with three columns can be seen in Table \ref{tab:requirements}.
%\begin{longtable}{|p{0,5cm}|p{10cm}|p{3cm}|}
%	\caption{\it{A table with some requirements}}
%	\label{tab:requirements}\\ \hline
%	\textbf{Nr} &  \textbf{Requirement} & \textbf{Weight}  \\
%	\hline
%	\endfirsthead
%	\multicolumn{3}{l}%
%	{\tablename\ \thetable\ -- \textit{Continues...}} \\
%	\hline
%	\textbf{Nr} &  \textbf{Requirement} & \textbf{Importance}  \\
%	\hline
%	\endhead
%	\hline \multicolumn{3}{l}{\textit{Continues...}} \\
%	\endfoot
%	\hline
%	\endlastfoot
%1 & Price & High\\ \hline
%2 & Variety& Middle\\ \hline
%3 & Support& Low\\ \hline
%
%%%\end{longtable}
%%
%We can use variables set in the \textit{main.tex} file to render values like our title (\doctitle) or supervisor names (\textbf{Supervisor}: \supervisor, \textbf{Co-supervisor}: \cosupervisor{}).

Lõputöö eesmärk oli luua rakendus, mis tõmbaks kasutaja andmed treeningute kohta Polar RC3 GPS kellast kasutaja arvutisse ning siis neid andmeid analüüsiks, võimaldaks seada eesmärke ja jälgida enda arengut.
Konkreetsemalt keskenduti andmete kättesaamisele kellast, kuna ei ole teada täpne protsess kella ja arvuti suhtlusel.
Rakendus oli planeeritud lihtsa käsurea programmina.

\section{Probleem}\label{sec:probleem}
Spordikella kasutades kogub see andmeid treeningu kohta.
Treeninguid on erinevaid mida saab salvestada, põhiliselt on kella fookus aeroobne treening, jooksmine ja rattasõit.
Põhilised andmed, mida kogutakse on läbitud distants, aeg, südametegevus ja muu sarnane.
Konkreetse kella mudel ise on vana ja tootja Polar seda ise enam ei toeta.
Seetõttu ei saa kasutada nende enda rakendusi kellast andmete kättesaamiseks ja nende analüüsimiseks.

!! kas raakida ka sellest et vana rakenduse toopohimote on vaga risune ja rumal lahendus? voi pigem kuskil sisus selle tausta lahti seletades? !!

\section{Lahendus}\label{sec:lahendus}
Leida ja analüüsida olemasolevaid sarnaseid lahendusi, ka tootja enda rakendust ja saada aru kella andmevahetusest, et luua toimiv rakendus, mida saaks kasutada ka siis, kui tootele tuge enam ei ole.
Rakenduse lähtekood jääb avatuks, et seda saaks vajadusel ka teised muuta ning funktsionaalsust laiendada.

